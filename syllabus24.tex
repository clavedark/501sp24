% Options for packages loaded elsewhere
\PassOptionsToPackage{unicode}{hyperref}
\PassOptionsToPackage{hyphens}{url}
\PassOptionsToPackage{dvipsnames,svgnames,x11names}{xcolor}
%
\documentclass[
  letterpaper,
  DIV=11,
  numbers=noendperiod]{scrartcl}

\usepackage{amsmath,amssymb}
\usepackage{iftex}
\ifPDFTeX
  \usepackage[T1]{fontenc}
  \usepackage[utf8]{inputenc}
  \usepackage{textcomp} % provide euro and other symbols
\else % if luatex or xetex
  \usepackage{unicode-math}
  \defaultfontfeatures{Scale=MatchLowercase}
  \defaultfontfeatures[\rmfamily]{Ligatures=TeX,Scale=1}
\fi
\usepackage{lmodern}
\ifPDFTeX\else  
    % xetex/luatex font selection
\fi
% Use upquote if available, for straight quotes in verbatim environments
\IfFileExists{upquote.sty}{\usepackage{upquote}}{}
\IfFileExists{microtype.sty}{% use microtype if available
  \usepackage[]{microtype}
  \UseMicrotypeSet[protrusion]{basicmath} % disable protrusion for tt fonts
}{}
\makeatletter
\@ifundefined{KOMAClassName}{% if non-KOMA class
  \IfFileExists{parskip.sty}{%
    \usepackage{parskip}
  }{% else
    \setlength{\parindent}{0pt}
    \setlength{\parskip}{6pt plus 2pt minus 1pt}}
}{% if KOMA class
  \KOMAoptions{parskip=half}}
\makeatother
\usepackage{xcolor}
\setlength{\emergencystretch}{3em} % prevent overfull lines
\setcounter{secnumdepth}{-\maxdimen} % remove section numbering
% Make \paragraph and \subparagraph free-standing
\ifx\paragraph\undefined\else
  \let\oldparagraph\paragraph
  \renewcommand{\paragraph}[1]{\oldparagraph{#1}\mbox{}}
\fi
\ifx\subparagraph\undefined\else
  \let\oldsubparagraph\subparagraph
  \renewcommand{\subparagraph}[1]{\oldsubparagraph{#1}\mbox{}}
\fi


\providecommand{\tightlist}{%
  \setlength{\itemsep}{0pt}\setlength{\parskip}{0pt}}\usepackage{longtable,booktabs,array}
\usepackage{calc} % for calculating minipage widths
% Correct order of tables after \paragraph or \subparagraph
\usepackage{etoolbox}
\makeatletter
\patchcmd\longtable{\par}{\if@noskipsec\mbox{}\fi\par}{}{}
\makeatother
% Allow footnotes in longtable head/foot
\IfFileExists{footnotehyper.sty}{\usepackage{footnotehyper}}{\usepackage{footnote}}
\makesavenoteenv{longtable}
\usepackage{graphicx}
\makeatletter
\def\maxwidth{\ifdim\Gin@nat@width>\linewidth\linewidth\else\Gin@nat@width\fi}
\def\maxheight{\ifdim\Gin@nat@height>\textheight\textheight\else\Gin@nat@height\fi}
\makeatother
% Scale images if necessary, so that they will not overflow the page
% margins by default, and it is still possible to overwrite the defaults
% using explicit options in \includegraphics[width, height, ...]{}
\setkeys{Gin}{width=\maxwidth,height=\maxheight,keepaspectratio}
% Set default figure placement to htbp
\makeatletter
\def\fps@figure{htbp}
\makeatother
\newlength{\cslhangindent}
\setlength{\cslhangindent}{1.5em}
\newlength{\csllabelwidth}
\setlength{\csllabelwidth}{3em}
\newlength{\cslentryspacingunit} % times entry-spacing
\setlength{\cslentryspacingunit}{\parskip}
\newenvironment{CSLReferences}[2] % #1 hanging-ident, #2 entry spacing
 {% don't indent paragraphs
  \setlength{\parindent}{0pt}
  % turn on hanging indent if param 1 is 1
  \ifodd #1
  \let\oldpar\par
  \def\par{\hangindent=\cslhangindent\oldpar}
  \fi
  % set entry spacing
  \setlength{\parskip}{#2\cslentryspacingunit}
 }%
 {}
\usepackage{calc}
\newcommand{\CSLBlock}[1]{#1\hfill\break}
\newcommand{\CSLLeftMargin}[1]{\parbox[t]{\csllabelwidth}{#1}}
\newcommand{\CSLRightInline}[1]{\parbox[t]{\linewidth - \csllabelwidth}{#1}\break}
\newcommand{\CSLIndent}[1]{\hspace{\cslhangindent}#1}

\KOMAoption{captions}{tableheading}
\makeatletter
\makeatother
\makeatletter
\makeatother
\makeatletter
\@ifpackageloaded{caption}{}{\usepackage{caption}}
\AtBeginDocument{%
\ifdefined\contentsname
  \renewcommand*\contentsname{Table of contents}
\else
  \newcommand\contentsname{Table of contents}
\fi
\ifdefined\listfigurename
  \renewcommand*\listfigurename{List of Figures}
\else
  \newcommand\listfigurename{List of Figures}
\fi
\ifdefined\listtablename
  \renewcommand*\listtablename{List of Tables}
\else
  \newcommand\listtablename{List of Tables}
\fi
\ifdefined\figurename
  \renewcommand*\figurename{Figure}
\else
  \newcommand\figurename{Figure}
\fi
\ifdefined\tablename
  \renewcommand*\tablename{Table}
\else
  \newcommand\tablename{Table}
\fi
}
\@ifpackageloaded{float}{}{\usepackage{float}}
\floatstyle{ruled}
\@ifundefined{c@chapter}{\newfloat{codelisting}{h}{lop}}{\newfloat{codelisting}{h}{lop}[chapter]}
\floatname{codelisting}{Listing}
\newcommand*\listoflistings{\listof{codelisting}{List of Listings}}
\makeatother
\makeatletter
\@ifpackageloaded{caption}{}{\usepackage{caption}}
\@ifpackageloaded{subcaption}{}{\usepackage{subcaption}}
\makeatother
\makeatletter
\@ifpackageloaded{tcolorbox}{}{\usepackage[skins,breakable]{tcolorbox}}
\makeatother
\makeatletter
\@ifundefined{shadecolor}{\definecolor{shadecolor}{rgb}{.97, .97, .97}}
\makeatother
\makeatletter
\makeatother
\makeatletter
\makeatother
\makeatletter
\@ifpackageloaded{fontawesome5}{}{\usepackage{fontawesome5}}
\makeatother
\ifLuaTeX
  \usepackage{selnolig}  % disable illegal ligatures
\fi
\IfFileExists{bookmark.sty}{\usepackage{bookmark}}{\usepackage{hyperref}}
\IfFileExists{xurl.sty}{\usepackage{xurl}}{} % add URL line breaks if available
\urlstyle{same} % disable monospaced font for URLs
\hypersetup{
  pdftitle={PLSC 501 Syllabus, spring 2024},
  colorlinks=true,
  linkcolor={blue},
  filecolor={Maroon},
  citecolor={Blue},
  urlcolor={Blue},
  pdfcreator={LaTeX via pandoc}}

\title{PLSC 501 Syllabus, spring 2024}
\author{}
\date{}

\begin{document}
\maketitle
\ifdefined\Shaded\renewenvironment{Shaded}{\begin{tcolorbox}[frame hidden, borderline west={3pt}{0pt}{shadecolor}, boxrule=0pt, interior hidden, breakable, enhanced, sharp corners]}{\end{tcolorbox}}\fi

\hypertarget{instructor}{%
\subsubsection{Instructor}\label{instructor}}

\begin{itemize}
\tightlist
\item
  \faIcon{user} ~ \href{https://clavedark.github.io}{Prof.~Dave Clark}
\item
  \faIcon{university} ~ LNG 57
\item
  \faIcon{envelope} ~ dclark@binghamton.edu
\item
  \faIcon{github} ~ \href{https://clavedark.github.io}{clavedark}
\end{itemize}

\hypertarget{ta}{%
\subsubsection{TA}\label{ta}}

\begin{itemize}
\tightlist
\item
  \faIcon{user} ~
  \href{https://www.linkedin.com/in/kathleen-bannon/}{Kathleen Bannon}
\item
  \faIcon{envelope} ~ kbannon1@binghamton.edu
\end{itemize}

\hypertarget{course-details}{%
\subsubsection{Course details}\label{course-details}}

\begin{itemize}
\tightlist
\item
  \faIcon{calendar-alt} ~ Spring 2024
\item
  \faIcon{calendar} ~ Wednesday
\item
  \faIcon{clock} ~ 9:40-12:40
\item
  \faIcon{location-dot} ~ LNG 332 Bing SSEL
\end{itemize}

\hypertarget{lab}{%
\subsubsection{Lab}\label{lab}}

\begin{itemize}
\tightlist
\item
  \faIcon{calendar} ~ Thursday
\item
  \faIcon{clock} ~ 1:15-2:15
\item
  \faIcon{location-dot} ~ CW 309
\end{itemize}

\hypertarget{seminar-description}{%
\subsection{Seminar Description}\label{seminar-description}}

This 4 credit hour seminar is about the principles of linear regression
models, focusing on the ordinary least squares approach to regression.
The course is data science oriented, emphasizing data managment,
wrangling, coding in \textbf{R}, and data visualization.

The class requires some background (provided in preview materials) in
probability theory, matrix algebra, and using \textbf{R}. A major goal
of the class is to provide the intuition behind how and why we do
empirical tests of theories of politics. Anyone can estimate a
statistical model, but not everyone can estimate and interpret and
informed, thoughful model. Understanding regression and linking
regression with theories of political behavior are key steps toward
saying insightful things about politics.

An important thing to realize is that it takes time and repetition to
really understand this stuff intuitively. Exposure to regression in
general (whether OLS, MLE, Bayes, etc.) by reading, hearing, and doing
over and over will make it stick. So don't get down or freaked out if
you feel like you don't get it. Instead, ask questions - of me, of the
TA, in class, in workshop, in office hours. Asking is essential.

The class meets one time per week for three hours. The required workshop
meets one time per week for 1 hour. My office hours are designed to be
homework help hours where I'll work in the grad lab with any of you who
are working on the exercises. The most productive pathway for this class
is for your to get in the habit of working together, and those office
hours are a good time for this.

\hypertarget{course-purpose}{%
\subsection{Course Purpose}\label{course-purpose}}

This seminar is a requirement in the Political Science Ph.D.~curriculum.
It is the second (of three) required quantitative methods courses aimed
at giving students a wide set of empirical tools to put to use in
testing hypotheses about politics. This seminar focuses on the method of
Ordinary Least Squares, and the linear model, emphasizing data science
skills.

\hypertarget{learning-objectives}{%
\subsection{Learning Objectives}\label{learning-objectives}}

At the end of the semester, students will be able to describe the linear
model, its assumptions, and interpretation. Students will be able to
manage data, estimate linear models, diagnose and correct models, and
generate quantities of interest. Students will be able to plot and
analyze their estimates and quantities of interest. Students will be
able to generate their own research designs, and test hypotheses using
the linear model.

\hypertarget{reading}{%
\subsection{Reading}\label{reading}}

The book for the class is:

\begin{itemize}
\tightlist
\item
  Wooldridge, J.M. 2013. \emph{Introductory Econometrics}. 5th edition,
  South-Western/Cengage. ISBN 978-1-111-53104-1.
\end{itemize}

I'm assigning an older edition of Wooldridge (the 5th) so cheaper copies
are easier to come by. If you choose an edition other than that, you'll
need to reconcile chapters, etc. Also, be warned that the cheaper
international edition (paperback) is alleged to differ in substantial
ways.

\hypertarget{this-is-the-most-important-section-of-the-syllabus}{%
\subsection{This is the most important section of the
syllabus}\label{this-is-the-most-important-section-of-the-syllabus}}

It will rarely be the case that we discuss the readings in seminar.
Their purposes are two-fold. First, the technical readings (Wooldridge)
are to frame the derivations of the models and techniques. These are
important because to understand how to interpret a model, you need a
technical understanding of its origins. This is not to say you need to
be able to derive these yourselves, or to perform complex mathematical
computations. It is to say that if you carefully read about the models
themselves, you will certainly find their interpretations easier. The
mantra for this course is this - interpretation is everything. Without
some understanding of how these models arise, you will find
interpretation hard.

Second, the applied readings (i.e.~articles) provide just that -
application in a political, economic, or social context, and application
in terms of interpretation of results. Some applications are nicely
done, some less so. Not only are these useful examples of what to do and
of what not to do, they will provide really useful models for your own
efforts to apply these statistical models. Interpretation is everything.

So reading is up to you, and is crucial. As much as I hate to say this,
if I get the sense at any point during the term you are not reading, I
will absolutely begin weekly reading quizzes; or perhaps require weekly
papers on the readings. Making them up will suck. Taking them will suck.
Grading them will suck. There really isn't much reading, and there's
just no excuse not to do it.

\hypertarget{how-to-read}{%
\subsection{How to Read}\label{how-to-read}}

Reading academic stuff is a bit different from most other reading.
Especially since you're reading large amounts in graduate school (and
the rest of your academic lives), it's important to think about what
you're trying to accomplish. With journal articles, the goal has to be
to understand the novel claim the paper makes, to understand why that
novel claim is interesting or novel (at least according to the author),
to understand how the author assesses the evidence, and to understand
what that evidence tells us. Most importantly, you should have two
answers to each of these - what the author says, and what you think. The
author may tell you the argument is important for some reason, and you
might disagree - you might think it's unimportant, or important for a
different reason, or wrong, or whatever.

Reading technical stuff is yet another category, but equally important.
The best way to read math models is to read them aloud (yes, out loud)
for three reasons. First, it forces you to slow down as you read it -
you talk slower than your eyes move. Second, it forces you to deal with
every element of the model - in silence, your mind will just skim right
over it to the next set of English words. Third, it engages two senses
reading silently does not - vocalization and hearing.

Reading out loud is not going to make everything crystal clear - but it
will open some pathways whereby math language isn't completely foreign.
Most importantly, doing so bridges the gap between the math and the
English accounts that normally follow, and make those English accounts
easier to make sense of.

\hypertarget{course-requirements-and-grades}{%
\subsection{Course Requirements and
Grades}\label{course-requirements-and-grades}}

The seminar requires the following:

\begin{itemize}
\tightlist
\item
  Problem sets - 50\% total
\item
  Replication/Extension project (including log, poster, etc.) - 50\%
\end{itemize}

Please note that all written assignments must be submitted as PDFs
either compiled in \(\LaTeX\) or in \textbf{R} markdown (Quarto).

You'll complete a series of problem sets, mostly applied. How many will
depend on how things move along during the term. Regarding the problem
sets - the work you turn in for the problem sets should clearly be your
own, but I urge you to work together - doing so is a great way to learn
and to overcome problems.

A word about completeness - attempt everything. To receive a passing
grade in the course, you must finish all elements of the course, so all
problem sets, all exams, papers, etc. To complete an element, you must
at least attempt all parts of the element - so if a problem set has 10
problems, you must attempt all 10 or the assignment is incomplete,
you've not completed every element of the course, and you cannot pass. I
realize there may be problems you have trouble with and even get wrong,
but you must try - the bottom line is don't turn in incomplete work.
Ever.

For technical and statistics training and help, Kathleen will lead a
\textbf{required workshop session every week, Thursday 1:15-2:15pm.}

The paper assignment asks you to replicate some published piece of
research (details on selecting this we'll discuss in seminar), to
comment critically on the theory and design as you replicate, and then
to build on the paper in a substantive way, extending the research. That
extension is the main part of the assignment - this is one of your first
major opportunities to develop a novel contribution to the empirical
literature. The replication/extension paper is due the Monday of exam
week. You'll keep a research log as you work on the paper during the
semester - this is due along with the paper, and is part of your total
grade. In addition, we'll have a poster session to present your
extensions at the end of the semester. Details to follow.

Grades will be assigned on the following scale:

\begin{longtable}[]{@{}
  >{\raggedright\arraybackslash}p{(\columnwidth - 6\tabcolsep) * \real{0.1111}}
  >{\centering\arraybackslash}p{(\columnwidth - 6\tabcolsep) * \real{0.1389}}
  >{\raggedright\arraybackslash}p{(\columnwidth - 6\tabcolsep) * \real{0.1111}}
  >{\centering\arraybackslash}p{(\columnwidth - 6\tabcolsep) * \real{0.1250}}@{}}
\toprule\noalign{}
\begin{minipage}[b]{\linewidth}\raggedright
Grade
\end{minipage} & \begin{minipage}[b]{\linewidth}\centering
Range
\end{minipage} & \begin{minipage}[b]{\linewidth}\raggedright
Grade
\end{minipage} & \begin{minipage}[b]{\linewidth}\centering
Range
\end{minipage} \\
\midrule\noalign{}
\endhead
\bottomrule\noalign{}
\endlastfoot
A & 94-100\% & C+ & 77-79\% \\
A- & 90--93\% & C & 73-76\% \\
B+ & 87--89\% & C- & 70-72\% \\
B & 83-86\% & D & 60-69\% \\
B- & 80-82\% & F & \textless60\% \\
\end{longtable}

\hypertarget{course-schedule}{%
\subsection{Course Schedule}\label{course-schedule}}

Referring to Wooldridge, 5th edition, 2013\}

Week 1. 17 Jan \textbf{Introduction, Regression Discussion }

\begin{itemize}
\tightlist
\item
  matrix notes, preview materials
\end{itemize}

Week 2, 24 Jan -- \textbf{Matrix/Bivariate Regression}

\begin{itemize}
\tightlist
\item
  Wooldridge, chapter 2, Appendix D, Appendix E
\end{itemize}

Week 3, 31 Jan -- \textbf{Implementing the Model}

\begin{itemize}
\tightlist
\item
  Wooldridge, chapter 3, Appendix E
\end{itemize}

Week 4, 7 Feb -- \textbf{Multivariate Regression, statistical control}

\begin{itemize}
\tightlist
\item
  Wooldridge, chapter 3, Appendix E
\end{itemize}

Week 5, 14 Feb -- \textbf{Inference in Regression}

\begin{itemize}
\tightlist
\item
  Wooldridge, chapter 4
\end{itemize}

Week 6, 21 Feb -- \textbf{Specifying the Model}

\begin{itemize}
\tightlist
\item
  Wooldridge, chapters 6 and 9
\item
  King (1986)
\end{itemize}

Week 7, 28 Feb -- \textbf{Dummy Variables \& Multiplicative
Interactions}

\begin{itemize}
\tightlist
\item
  Wooldridge, chapter 7
\item
  Brambor, Clark, and Golder (2006)
\end{itemize}

Week 8, 6 March -- \textbf{Spring Break}

Week 9, 13 March -- \textbf{Interactions (continued)}

\begin{itemize}
\tightlist
\item
  Wooldridge, chapter 7
\item
  Brambor, Clark, and Golder (2006)
\end{itemize}

Week 10, 20 March -- \textbf{Limited Dependent Variables}

\begin{itemize}
\tightlist
\item
  Wooldridge, chapter 7.5, and 17.1
\end{itemize}

Week 11, 27 March -- \textbf{Panels, Fixed \& Random Effects} -
Wooldridge, chapters 13, 14

Week 12, 3 April-- \textbf{Time Series}

\begin{itemize}
\tightlist
\item
  Wooldridge, chapters 10, 12
\end{itemize}

Week 13, 10 April -- \textbf{Non-constant Variance}

\begin{itemize}
\tightlist
\item
  Wooldridge, chapter 8
\end{itemize}

Week 14, 17 April -- \textbf{IV models}

\begin{itemize}
\tightlist
\item
  Wooldridge, chapter 15
\end{itemize}

Week 15, 24 April -- \textbf{Passover, no classes}

Week 16, 1 May -- \textbf{Causal Inference}

\begin{itemize}
\tightlist
\item
  TBA
\end{itemize}

\hypertarget{refs}{}
\begin{CSLReferences}{1}{0}
\leavevmode\vadjust pre{\hypertarget{ref-brambor2006understanding}{}}%
Brambor, T., W. R. Clark, and M. Golder. 2006. {``Understanding
Interaction Models: Improving Empirical Analyses.''} \emph{Political
Analysis} 14 (1): 63--82.

\leavevmode\vadjust pre{\hypertarget{ref-king1986lie}{}}%
King, Gary. 1986. {``How Not to Lie with Statistics: Avoiding Common
Mistakes in Quantitative Political Science.''} \emph{American Journal of
Political Science} 30 (3): 666--87.

\end{CSLReferences}



\end{document}
